\documentclass[12pt,letterpaper,cm]{hmcpset}
\usepackage[margin=1in]{geometry}
\usepackage{graphicx}
\usepackage{amsmath,amssymb}
\usepackage{enumerate}

% info for header block in upper right hand corner
\name{Gary Luo}
\class{Math 70}
\assignment{Hw 01}
\duedate{February 4, 2018}
\setlength\parindent{0pt}

\newcommand\Z{\mathbb{Z}}

\begin{document}

\problemlist{A1,A2,A3,A4,A5}

\begin{problem}[A1]
    \begin{enumerate}
        \item Find permutations $p,q$ in $S_3$ such that $p\cdot q \neq q\cdot p$
        \item Prove that if $n\geq 3$ then there exists $p,q \in S_n$ such that $p\cdot q \neq q\cdot p$
        \item Let $a,b\in J_n$ with $a\neq b$. Prove that $(a,b) = (b,a)$
        \item Prove that $(a,b)(a,b) = e$. Here $e$ denotes the identity in $S_n$
        \item Compute $(a,b,c)\cdot(a,b,c)\cdot(a,b,c)$.
    \end{enumerate}
 \end{problem}

\begin{solution}
    \begin{enumerate}
        \item Let $p=(1,2,3)$ and $q=(1,3)$. Then, we see that $p\cdot q = (1,2,3)$ and $q\cdot p = (1,2)$. Clearly, these are not the same.
        \item We will prove this by induction. For our base case, we will look at part 1 and see that it holds true for $n=3$. Next, 
    \end{enumerate}
\end{solution}

\pagebreak
%-----------------------------------------------------------------------------------------------------------------------%

\begin{problem}[A2]
    \begin{enumerate}
        \item Let $i\in J_n$. Prove that the permutations $(i,i+3)$ may be written as
        \begin{center}
            $(i,i+1)\cdot(i+1,i+2)\cdot(i+2,i+3)\cdot(i+2,i+1)\cdot(i+1,i)$.\\
        \end{center}
        Hint: Consider $x<i$, $i\leq x \leq i+3$ and $x>i+3$.
        \item Prove that if $c_i$ and $c_2$ are disjoint cycles then $c_1\cdot c_2 = c_2 \cdot c_1$
    \end{enumerate}
\end{problem}

\begin{solution}

\end{solution}

\pagebreak
%-----------------------------------------------------------------------------------------------------------------------%

\begin{problem}[A3]
    Prove that if $A=(a_{ij})$ is an $n \times n$ triangular matrix (either $a_{ij} = 0$ for all $i<j$ or $a_{ij}=0$ for all $i>j$), then $det(A) = a_{11}\cdot a_{22} \cdots a_{nn}$.
\end{problem}

\begin{solution}
    
\end{solution}

\pagebreak
%-----------------------------------------------------------------------------------------------------------------------%

\begin{problem}[A4]
    \begin{enumerate}
        \item Let $A,B,C$ be $2\times 2$ matrices. Let X be the $4\times 4$ matrix whose first two rows are $[A\ B]$ and its last two rows are $[0\ C]$, where 0 stands for the $2\times 2$ matrix whose entries are the all equal to zero. Prove that $det(X) = det(A)det(C)$.
        \item Generalize this result for $2n \times 2n$ matrices.
    \end{enumerate}
\end{problem}
\begin{solution}

\end{solution}
\pagebreak
%-----------------------------------------------------------------------------------------------------------------------%

\begin{problem}[A5]
    For each $i,j\in J_n$ let $a_{ij}(x)$ be a differentiable functions of x. Let $A(x) = (a_{ij}(x))$. Prove that $(det(A(x)))' = \summation_{i=1}^n det(B_i(x))$ where $B_i(x)$ is the matrix resulting from the replacing in $A(x)$ the i-th column by the column made up by the derivatives of the i-th column of $A(x)$. 
\end{problem}

\begin{solution}
    
\end{solution}

\end{document}
